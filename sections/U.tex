\newpage
	\flushright{\hyperref[index]{\color{black!65}{Ritorna all'indice}}}\flushleft
	\section{U} \label{sec:U}		
	
		\subsection{UNITÀ}	\index{Unità} \label{unita}	
		Divide l'organizzazione del lavoro di programmazione fino al punto in cui non conviene più (corrisponde ad una funzionalità e può essere per esempio una classe o dei dati). Ciò che ho prodotto si chiamano \underline{\hyperref[moduli]{moduli}}. Un unità può essere anche un insieme di più moduli e la specifica di ogni unità architetturale deve essere documentata.
	
		\subsection{USABLE}	\index{Usable}	\label{usable}
		Milestone usable, ovvero il sistema è utilizzare, ha le caratteristiche desiderate, infatti può essere operato dagli utenti. Le funzionalità e le prestazioni richieste sono state verificate e validate e la quantità di difetti è accettabile. Si rivolge alla \underline{\hyperref[productbaseline]{Product Baseline}}.
	
	
