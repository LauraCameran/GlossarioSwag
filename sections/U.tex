\newpage
	\section{U} \label{sec:U}

		\subsection{UNITÀ}	\index{Unità} \label{unita}
		Dividere l'organizzazione del lavoro di programmazione fino al punto in cui non conviene più, ovvero in cui si giunge all'unità.
		Essa corrisponde ad una funzionalità e può essere per esempio una classe o dei dati. \\
		Ciò che ho prodotto si chiamano \underline{\hyperref[moduli]{moduli}}.
		Un'unità infatti può essere anche un insieme di più moduli. \\
		La specifica di ogni unità architetturale deve essere documentata.

		\subsection{USABLE}	\index{Usable}	\label{usable}
		\textit{Milestone usable}, ovvero il sistema è utilizzabile, ha le caratteristiche desiderate, e infatti può essere operato dagli utenti. Le funzionalità e le prestazioni richieste sono state verificate e validate e la quantità di difetti è accettabile. \\
		Si rivolge alla \underline{\hyperref[productbaseline]{Product Baseline}} ed è uno degli stati di progresso del \underline{\hyperref[semat]{SEMAT}} per la \underline{\hyperref[progettazione]{progettazione}}.
