\newpage
	\section{R} \label{sec:R}

		\subsection{READY}	\index{Ready}	\label{ready}
		Vuol dire che nessuna parte manca e la documentazione è pronta. Gli stakeholders hanno accettato il prodotto e vogliono che diventi operativo. \\
		È l'ultimo degli stati di progresso del \underline{\hyperref[semat]{SEMAT}} per la \underline{\hyperref[progettazione]{progettazione}} (vedi \underline{\hyperref[solutionimage]{immagine}}).

		\subsection{REALIZZAZIONE} \index{Realizzazione}	\label{realizzazione}
		Attua ciò che è stato progettato durante la \underline{\hyperref[progettazione]{progettazione software}}: avviene quindi la stesura del codice SENZA INVENTARE. Infine avviene l'attività di \underline{\hyperref[collaudo]{collaudo}} che è il nostro passo di uscita sapendo l'esito che avrà.

		\subsection{REQUIREMENTS} \index{Requirements} \label{requirements}
		Idea di soluzione per soddisfare il bisogno (quindi astratta).
		È la descrizione documentata di una \textit{capacità}
		\begin{itemize}
			\item Necessaria a un utente per raggiungere un obiettivo (vista dal lato del cliente)
			\item Che un sistema deve possedere per adempiere a un obbligo (vista dal lato dello sviluppatore)
		\end{itemize}
	    I requisiti devono essere tutti identificabili e verificabili. Vengono \underline{\hyperref[classificazione]{classificati}}, per facilitare la comprensione, la \underline{\hyperref[manutenzione]{manutenzione}} e il \underline{\hyperref[tracciamento]{tracciamento}}, in base ad
	    	\begin{itemize}
	    		\item \textbf{Attributi di prodotto}: rispondono a \textit{Cosa devo fare?} (requisiti funzionali, prestazionali e di qualità)
	    		\item \textbf{Attributi di processo}: rispondono a \textit{Come devo farlo?} (requisiti di vincolo)
	    	\end{itemize}
		Ogni tipo di requisito può essere diretto o indiretto, implicito o esplicito ma ognuno deve essere \underline{\hyperref[verificare]{verificabile}} tramite:
		\begin{itemize}
			\item \textbf{Per requisiti funzionali}: test, dimostrazione frontale, revisione
			\item \textbf{Per requisiti prestazionali}: misurazioni
			\item \textbf{Per requisiti qualitativi}: tecniche ad hoc
			\item \textbf{Per requisiti dichiarativi (vincoli)}: revisione
		\end{itemize}
		In base all'utilità strategica possono inoltre essere classificati in:
		\begin{itemize}
			\item \textbf{Obbligatori}: irrinunciabili secondo gli \underline{\hyperref[stakeholder]{stakeholders}}
			\item \textbf{Desiderabili}: non strettamente necessari ma danno valore aggiunto
			\item \textbf{Opzionali}: relativamente utili
		\end{itemize}
		Questo perché devo capire quali requisiti sono irrinunciabili. I requisiti sono quindi elastici e dipendono da me, da quanto sono veloce, ecc.

		\underline{\hyperref[qualita]{Qualità}} che ci devono essere secondo \underline{\hyperref[ieee830]{IEEE 830}}:
		\begin{itemize} %slide 25/32
			\item \textbf{Non ambiguo} ovvero il requisito deve essere scritto in maniera concisa per non creare confusione (ma attenzione perché non c'è una metrica)
			\item \textbf{Corretto}
			\item \textbf{Completo} perché c'è tutto ciò che ci deve essere
			\item \textbf{Verificabile}
			\item \textbf{Consistente} perché non possono essere contraddittori
			\item \textbf{Modificabile} con la \underline{\hyperref[gestionecambiamenti]{gestione dei cambiamenti}}, ma l'indice che ogni requisito ha non lo do io (altrimenti se modifico qualcosa devo riordinare tutto) e deve essere significativo
			\item \textbf{Tracciabile}
			\item \textbf{Ordinato per prevalenza}
		\end{itemize}

		Durante l'\underline{\hyperref[analisideirequisiti]{analisi dei requisiti}} avviene inoltre la \underline{\hyperref[verificare]{verifica}} dei requisiti:
			\begin{itemize}
				\item Viene eseguita su un documento organizzato tramite \underline{\hyperref[walkthrough]{walkthrough}} o \underline{\hyperref[inspection]{ispezione}} (lettura mirata)
				\item Si ricerca chiarezza espressiva, che non è il linguaggio naturale
				\item Si ricerca chiarezza strutturale, ovvero si controlla la separazione tra requisiti funzionali e non-funzionali e che la \underline{\hyperref[classificazione]{classificazione dei requisiti}} sia precisa, uniforme e accurata
				\item Si ricerca atomicità e aggregazione, quindi i requisiti elementari e le correlazioni tra di essi chiare ed esplicite
			\end{itemize}
		E anche la \underline{\hyperref[gestionerequisiti]{gestione dei requisiti}}.


		\subsection{RESPONSABILE} \index{Responsabile} \label{responsabile}
		È uno dei \underline{\hyperref[ruoli]{ruoli}} in un progetto. Partecipa al \underline{\hyperref[progetto]{progetto}} per tutta la sua durata rappresentando il progetto presso il fornitore e il committente. Accentra le responsabilità di scelta e approvazione, quindi deve avere conoscenze e capacità tecniche per valutare (rischi e scelte varie). Ha molte responsabilità, nello specifico su: \underline{\hyperref[pianificazione]{pianificazione}}, gestione delle risorse umane, controllo, coordinamento e relazioni con l'esterno.


		\subsection{REVISIONE DEI REQUISITI} \index{Revisione dei Requisiti} \label{RR}
		Revisione 1. \\
		Ha la funzione di concordare con il cliente una visione condivisa del prodotto atteso.
		Prevede:
		\begin{itemize}
			\item \underline{\hyperref[analisideirequisiti]{Analisi dei Requisiti}}
			\item \underline{\hyperref[piano]{Piano di Progetto}}
			\item \underline{\hyperref[pianoqualifica]{Piano di Qualifica}}
			\item \underline{\hyperref[norme]{Norme di Progetto}}
			\item \underline{\hyperref[studiofattibilita]{Studio di Fattibilità}}
		\end{itemize}
		Fa l'\underline{\hyperref[audit]{Audit Process}}.


		\subsection{REVISIONE DI ACCETTAZIONE} \index{Revisione di Accettazione} \label{RA}
		Revisione 4. \\
		Ha la funzione di collaudare il sistema per accettazione da parte del committente e accertarsi del soddisfacimento di tutti i requisiti utente fissati nella RR.
		Prevede:
		\begin{itemize}
			%\item \underline{\hyperref[productbaseline]{Product Baseline}}
			\item \underline{\hyperref[pianoqualifica]{Piano di Qualifica}} (quarta versione)
			\item \underline{\hyperref[piano]{Piano di Progetto}}(quarta versione) con \underline{\hyperref[consuntivo]{consuntivo}} finale
			\item Manuale Utente e Manuale Sviluppatore (entrambe seconda versione)
		\end{itemize}
		Fa l'\underline{\hyperref[audit]{Audit Process}}.


		\subsection{REVISIONE DI PROGETTAZIONE}	\index{Revisione di Progettazione} \label{RP}
		Revisione 2. \\
		Ha la funzione di accertare la realizzabilità.
		Prevede:
		\begin{itemize}
			\item \underline{\hyperref[technologybaseline]{Technology Baseline}}
			\item \underline{\hyperref[pianoqualifica]{Piano di Qualifica}} (seconda versione)
			\item \underline{\hyperref[piano]{Piano di Progetto}} (seconda versione)
			\item \underline{\hyperref[norme]{Norme di Progetto}} (seconda versione)
		\end{itemize}
		Fa il \underline{\hyperref[joint]{Joint Review Process}}.


		\subsection{REVISIONE DI QUALIFICA} \index{Revisione di Qualifica} \label{RQ}
		Revisione 3. \\
		Ha la funzione di approvare l’esito finale delle verifiche e attivare la validazione.
		Prevede:
		\begin{itemize}
			\item \underline{\hyperref[productbaseline]{Product Baseline}}
			\item \underline{\hyperref[pianoqualifica]{Piano di Qualifica}} (terza versione)
			\item \underline{\hyperref[piano]{Piano di Progetto}} (terza versione)
			\item Manuale Utente e Manuale Sviluppatore
		\end{itemize}
		Fa il \underline{\hyperref[joint]{Joint Review Process}}.


		\subsection{RIUSO} \index{Riuso} \label{riuso}
		Il riuso può essere:
			\begin{itemize}
				\item \textbf{Occasionale}: perchè ``copia e incolla'' opportunistico, che ha quindi un basso costo, ma scarso impatto
				\item \textbf{Sistematico}: perchè è un ``copia e incolla'' intelligente, che ha un maggior costo ma maggior impatto
			\end{itemize}
		Nel breve periodo il riuso diventa più che altro un costo.


		\subsection{RUOLI} \index{Ruoli} \label{ruoli}
		Il ruolo è una persona che si occupa di una \textit{funzione aziendale} assegnata in un progetto. Tra queste funzioni aziendali troviamo:
		\begin{itemize}
			\item \textbf{Sviluppo}: quindi la persona dovrà avere responsabilità tecnica e realizzativa
			\item \textbf{Direzione}: quindi la persona dovrà avere responsabilità decisionale.
			\item \textbf{Amministrazione}: ovvero gestione del supporto ai processi.
			\item \textbf{\underline{\hyperref[qualita]{Qualità}}}: ovvero gestione della ricerca di economicità.
		\end{itemize}
		A seguire, l'elenco dei ruoli all'interno della \underline{\hyperref[gestioneprogetto]{gestione del progetto}} (ricordiamo che non vogliamo sovrapposizioni):
		\begin{itemize}
		\item \textbf{\underline{\hyperref[analista]{Analista}}}: fa l'analisi dei requisiti. Capisce ciò di cui c'è bisogno nel mio prodotto, per cui è importante che mettersi nei panni dell'utente.
		\item \textbf{\underline{\hyperref[progettista]{Progettista}}}: pensa ad una possibile soluzione con qualità desiderabili (\textbf{NB}: non implementa).
		\item \textbf{\underline{\hyperref[programmatore]{Programmatore}}}: concretizza la soluzione che il progettista ha pensato (\textbf{NB}: non inventa).
		\item \textbf{\underline{\hyperref[verificatore]{Verificatore}}}: dice se il lavoro del programmatore è conforme alle attese. Deve poi relazionarsi ed essere di supporto.
		\item \textbf{\underline{\hyperref[responsabile]{Responsabile}}}: ha ``onere e onore'', in particolare ha tanti oneri. Coordina il team e garantisce che non ci siano intoppi o rallentamenti.
		\item \textbf{\underline{\hyperref[amministratore]{Amministratore}}}: garantisce che funzioni bene il nostro sistema informatico (che sia scalabile, ecc).
		\end{itemize}
