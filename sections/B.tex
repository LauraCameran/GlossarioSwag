\newpage
	\flushright{\hyperref[index]{\color{black!65}{Ritorna all'indice}}}\flushleft
	\section{B} \label{sec:B}
	
		\subsection{BACKLOG} \index{Backlog} \label{backlog}
		(Chiamato anche "to do") Cose da fare per soddisfare la \textit{user story}, ovvero l'informale descrizione delle caratteristiche che il prodotto software deve avere.
		
		
		\subsection{BASELINE} \index{Baseline}  \label{baseline} 
		Letteralmente "campo base" o "punto d'appoggio" (per evitare situazioni di rischio). È un risultato concreto che risponde alla domanda:"Sì, si può fare". È una risposta buona e possibile in quel dato momento.
		\begin{itemize}
			\item \textbf{Cos'è}: Un progetto software, se si basa su una strategia, avrà una sequenza di obiettivi. Le parti di cui è fatta una baseline (le quali hanno un numero di versioni "as many as needed"), esistono perché assolvono un obiettivo. Una baseline rappresenta quindi un punto di avanzamento consolidato in un dato istante del progetto. Ogni punto di avanzamento viene fissato precedentemente in modo strategico dalla \underline{\hyperref[best]{best practice}}, ma il numero di baseline non è deciso a priori (solo gli obiettivi sono decisi a priori). 
			\item \textbf{A cosa serve}: Dare una base da cui partire per l'avanzamento del progetto.
			\item \textbf{Come si mantiene}: Bisogna decidere come le parti concorrono a formare la baseline tramite versionamento.
			\item \textbf{Come si costruisce}: Tramite configurazione.
		\end{itemize}
		%Ci possono essere più baseline ed essere consecutive. 
		%Insieme di CI (parti che compongono il prodotto che possegono ID, nome, data, autore, registro delle modifiche, stato corrente) consolidato a un dato istante (milestone/data di calendario). 
		
		
		\subsection{BEST PRACTICE} \index{Best practice} \label{best}
		\underline{\hyperref[way]{Modo di fare}} che deve garantire i migliori risultati in specifiche e note circostanze.
		
		
		\subsection{BODY OF KNOWLEDGE} \index{Body of knowledge} \label{body}
		"Corpo"/Insieme di conoscenze che ci ha emancipato.
		
		
		\subsection{BOTTOM-UP} \label{bottomup}
		Concepisco il sistema basandomi dalle ipotetiche parti che possono comporlo. Questo approccio è tipico della \textit{Programmazione ad Oggetti} (esempio di: costruisco il frigo sapendo che esso è un aggregato di reparti). 
		
		
		\subsection{BRAINSTORMING} \index{Brainstorming} \label{brainstorming}
		Pensiero intenso collettivo che fa nascere idee, in cui ognuno parla a turno e non c'è sopraffazione. Durante la discussione, una sola persona scrive. È uno strumento utile per unire debolezze in un'unica forza maggiore.
	
	
		\subsection{BRANCH COVARAGE}	\index{Branch Coverage}	\label{branchcoverage}
		Si occupa di coprire i rami di decisione. Si ha quindi copertura al 100\% quando ciascun ramo del flusso di controllo dell'unità viene attraversato almeno uno volta, ognuno con esito corretto. \\
		Il valore di copertura è determinato dalla complessità di espressioni di decisione (espressioni composte da condizioni contenenti valori booleani).
		%Quando compongo condizioni con operazioni abbiamo Decisione(in rosso). %sono un genio
		