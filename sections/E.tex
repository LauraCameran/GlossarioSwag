\newpage
	\section{E} \label{sec:E}

		\subsection{ECONOMICITÀ} \index{Economicità} \label{economicita}
		L'insieme di \underline{\hyperref[efficienza]{efficienza}} ed \underline{\hyperref[efficacia]{efficacia}}.

		\subsection{EFFICACIA} \index{Efficacia} \label{efficacia}
		Capacità di produrre l'effetto e i \textbf{risultati voluti}. Misurabile grazie al \underline{\hyperref[pianoqualita]{Piano di Qualità}}. (Il termine è in stretta correlazione a \underline{\hyperref[qualita]{qualità}} e conformità).


		\subsection{EFFICIENZA} \index{Efficienza} \label{efficienza}
		Capacità costante di rendimento e rispondenza per i propri fini \textbf{senza sprecare risorse}. \\
		Metrica di riferimento: \underline{\hyperref[produttivita]{produttività}}.
		Si ha nel \underline{\hyperref[piano]{Piano di Progetto}}, perchè misuro spartendo le risorse.

		\subsection{ENDEAVOR} \index{Endeavor} \label{endeavor}
		Atto costruttivo di fare (= tentativo/provare) comprende
		\begin{itemize}
			\item \textbf{Work}: attività e compiti da fare \label{work}
			\item \textbf{Team}: perché collaborativo \label{team}
			\item \textbf{Way of working}: un team è un team soltanto se ha il suo modo di lavorare, cosa che sta alla base di tutto e regola i \underline{\hyperref[processo]{processi}} \label{way}
		\end{itemize}
			%\subsubsection{} \label{work}
			%\subsubsection{} \label{team}
			%\subsubsection{Way of working}	\label{way}


		\subsection{ENGINEERING} \index{Engineering} \label{engineering}
		Applicazione pratica di principi scientifici e matematici. Ciò vuol dire che non \textit{crea} bensì \textit{attua} secondo la \underline{\hyperref[best]{best practice}}. Comporta inoltre responsabilità etiche e professionali.

		\subsection{ERROR}	\index{Error}	\label{error}
		Ha a che vedere con la gerarchia di problemi nei \underline{\hyperref[test]{test}}. \\
		Error è a monte della \underline{\hyperref[failure]{failure}} e accade perché lo stato del sistema è sbagliato. Può essere meccanico, algoritmico o concettuale.
