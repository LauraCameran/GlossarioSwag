\newpage
	\section{E} \label{sec:E}
	
		\subsection{ECONOMICITÀ} \index{Economicità} \label{economicita}
		L'insieme di \underline{\hyperref[efficienza]{efficienza}} ed \underline{\hyperref[efficacia]{efficacia}}.
		
		\subsection{EFFICACIA} \index{Efficacia} \label{efficacia}
		Capacità di produrre l'effetto e i risultati voluti. (In stretta correlazione a \underline{\hyperref[qualita]{qualità}} e conformità).
		Ci misuriamo grazie al Piano di Qualità.
		
		\subsection{EFFICIENZA} \index{Efficienza} \label{efficienza}
		Capacità costante di rendimento e rispondenza per i propri fini senza sprecare risorse. Metrica di riferimento: \underline{\hyperref[produttivita]{produttività}}.
		Si ha nel Piano di Progetto, perchè misuro spartendo le risorse. 
		
		\subsection{ENDEAVOR} \index{Endeavor} \label{endeavor}
		Atto costruttivo di fare (=tentativo/provare) comprende:
			\subsubsection{Work} \label{work}
			Attività e compiti da fare.
			\subsubsection{Team} \label{team}
			Perché collaborativo.
			\subsubsection{Way of working}	\label{way}
			Un team è un team soltanto se ha il suo modo di lavorare, cosa che sta alla base di tutto. Esso regola i \underline{\hyperref[processo]{processi}}.
		
		\subsection{ENGINEERING} \index{Engineering} \label{engineering}
		Applicazione pratica di principi scientifici e matematici. Ciò vuol dire che non \textit{crea} bensì \textit{attua} secondo la \underline{\hyperref[best]{best practice}}. Comporta inoltre responsabilità etiche e professionali.
		
		\subsection{ERROR}	\index{Error}	\label{error}
		È a monte della \underline{\hyperref[failure]{failure}} e accade perché lo stato del sistema è sbagliato. Può essere meccanico, algoritmico o concettuale.
	
	
