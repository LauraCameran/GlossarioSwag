\newpage
	\section{F} \label{sec:F}
	
		\subsection{FAILURE}	\index{Failure}	\label{failure}	%V e V: analisi dinamica 20 dicembre
		L'effetto che vedo di un malfunzionamento. È un effetto finale che non mi aspettavo. \\
		In una versione gerarchica, più grande, un failure potrebbe generare un altro \underline{\hyperref[fault]{fault}} che causa altri errori. %slide 6/34
	
		\subsection{FASE} \index{Fase} \label{fase}  
		Segmento contiguo di durata temporale che ha inizio e fine. É diviso a sua volta in altri segmenti. Può essere:
		\begin{itemize}
			\item di \underline{\hyperref[stato]{stato}};
			\item di transizione (stato che mi porta da x a y).
		\end{itemize}
	
		\subsection{FAULT}	\index{Fault}	\label{fault}
		È la ragione dell'\underline{\hyperref[error]{error}}, causa l'errore. Viene da un \underline{\hyperref[mistake]{mistake}} umano.
		
	
	
