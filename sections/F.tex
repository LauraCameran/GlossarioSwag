\newpage
	\section{F} \label{sec:F}

		\subsection{FAILURE}	\index{Failure}	\label{failure}	%V e V: analisi dinamica 20 dicembre
		Ha a che vedere con la gerarchia di problemi nei \underline{\hyperref[test]{test}}. \\
		Failure è l'effetto che vedo di un malfunzionamento, un effetto finale che non mi aspettavo. \\
		In una versione gerarchica più grande, un failure potrebbe generare un altro \underline{\hyperref[fault]{fault}} che causa altri errori. %slide 6/34

		\subsection{FASE} \index{Fase} \label{fase}
		Segmento contiguo di durata temporale che ha inizio e fine. È diviso a sua volta in altri segmenti. Può essere:
		\begin{itemize}
			\item Di \underline{\hyperref[stato]{stato}}
			\item Di transizione (stato che mi porta da x a y)
		\end{itemize}

		\subsection{FAULT}	\index{Fault}	\label{fault}
		Ha a che vedere con la gerarchia di problemi nei \underline{\hyperref[test]{test}}. \\
		È la ragione dell'\underline{\hyperref[error]{error}}, ovvero causa l'errore. Proviene da un \underline{\hyperref[mistake]{mistake}} umano.
