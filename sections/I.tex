\newpage
	\flushright{\hyperref[index]{\color{black!65}{Ritorna all'indice}}}\flushleft
	\section{I} \label{sec:I}
	
		\subsection{IEEE 830-1998}	\index{IEEE 830-1998} \label{ieee830}
		\textit{Recommended Practice for Software Requirements Specifications} dice che la specifica deve essere:
			\begin{itemize}
				\item Priva di ambiguità;
				\item Corretta;
				\item Completa;
				\item Verificabile;
				\item Consistente;
				\item Modificabile;
				\item Tracciabile;
				\item Ordinata per rilevanza;
			\end{itemize}
		Per la restante documentazione vedi \href{https://www.cs.purdue.edu/homes/apm/courses/BITSC461-fall03/miller-guidelines/IEEE830-1998.html}{qui}.
	
		\subsection{IMPLEMENTATION} \index{Implementation} \label{implementation}
		(Codifica) Realizzo concretamente quello che ho pensato.
		
		\subsection{INCREMENTO} \index{Incremento} \label{incremento}
		Modo di avvicinarsi sempre di più a destinazione (aggiungendo qualcosa che mi porta verso la meta).
		
		\subsection{INSPECTION}	\index{Inspection}	\label{inspection}
		È un metodo pratico di lettura come \underline{\hyperref[walkthrough]{Walkthrough}}. Ha come obiettivo rilevare la presenza di difetti eseguendo una lettura mirata. Chi la fa sono verificatori distinti dai programmatori e la strategia che viene adottata si focalizza sulla ricerca con presupposti. \\
		Anche qui in ogni fase delle attività svolte deve avvenire la documentazione:
		\begin{itemize}
			\item fase 1: pianificazione;
			\item fase 2: definizione lista di controllo;
			\item fase 3: lettura;
			\item fase 4: correzione dei difetti;
		\end{itemize}
		\textbf{Differenze}: Walkthrough richiede maggiore attenzione ed è più collaborativo, mentre Inspection è più rapido e basato su presupposti.
		
		\subsection{INTEGRAZIONE CONTINUA} \index{Integrazione continua} \label{integrazione}
		Incremento e non itero (in modo dimostrabile), vado quindi verso la meta. (Ogni passo che eseguo ha un valore).
		
		
		\subsection{ITERAZIONE} \index{Iterazione} \label{iterazione}
		Procedere per raffinamenti o rivisitazioni (aggiungendo o togliendo qualcosa che mi fa quindi avanzare o retrocedere).
	
	
	
