\newpage
	\section{I} \label{sec:I}


		\subsection{IMPLEMENTATION} \index{Implementation} \label{implementation}
		Termine inglese per \textit{codifica}. Si tratta di realizzare concretamente quello che nel progetto è stato precedentemente ideato.


		\subsection{INCAPSULAMENTO} \index{Incapsulamento} \label{incapsulamento}
		Un meccanismo del linguaggio di programmazione atto a limitare l'accesso diretto agli elementi dell'oggetto.


		\subsection{INCREMENTO} \index{Incremento} \label{incremento}
		Modo di avvicinarsi sempre di più a destinazione ``aggiungendo qualcosa che mi porta verso la meta''.


		\subsection{INSPECTION}	\index{Inspection}	\label{inspection}
		È un metodo pratico di lettura come \underline{\hyperref[walkthrough]{Walkthrough}}. Ha come obiettivo rilevare la presenza di difetti eseguendo una lettura mirata. Chi la fa sono \underline{\hyperref[verificatore]{Verificatori}} distinti dai \underline{\hyperref[programmatore]{Programmatori}} e la strategia che viene adottata si focalizza sulla ricerca con presupposti. \\
		In ogni fase delle attività svolte deve avvenire la documentazione:
		\begin{enumerate}
			\item Pianificazione
			\item Definizione lista di controllo
			\item Lettura
			\item Correzione dei difetti
		\end{enumerate}
		\textbf{Differenze}: Walkthrough richiede maggiore attenzione ed è più collaborativo, mentre Inspection è più rapido e basato su presupposti.


		\subsection{INTEGRAZIONE CONTINUA} \index{Integrazione continua} \label{integrazione}
		In modo dimostrabile \underline{\hyperref[incremento]{Incremento}} e non itero. Vado quindi verso la meta e ogni passo che eseguo ha un valore.


		\subsection{ITERAZIONE} \index{Iterazione} \label{iterazione}
		Procedere per raffinamenti o rivisitazioni ``aggiungendo o togliendo qualcosa che mi fa quindi avanzare o retrocedere''.
