\newpage
	\section{L} \label{sec:L}

		\subsection{LAZY} \index{Lazy} \label{lazy}
		\textit{Lazy evaluation} è una strategia di valutazione che \textit{delays the evaluation of an expression until its value is needed} (non-strict evaluation) e che inoltre evita valutazioni ripetute (sharing).


		%26 febbraio 2019 - Misurazione del software - Metodi e obiettivi di quantificazione
		\subsection{LEAD TIME}	\index{Lead time} \label{leadtime}
		%slide 15/25
		È quanto tempo trascorre da quando un'azione è assegnata a quando questa è completata. Riesco a capire che una persona non sta svolgeno il proprio lavoro se il tempo medio aumenta. Se stiamo finendo invece, il valore tenderà a zero. \\
		È una \underline{\hyperref[metrica]{metrica}} di \underline{\hyperref[gestioneprogetto]{project management}}.


		\subsection{LOGGER}	\index{Logger}	\label{logger}
		Componente non intrusivo di registrazione dei dati di esecuzione per l'analisi dei risultati, quale per esempio un \textit{bot}. \\
		È uno dei metodi di automatizzazione per i \underline{\hyperref[test]{test}}.
