\newpage
	\section{Q} \label{sec:Q}

		\subsection{QUALIFICA} \index{Qualifica} \label{qualifica}
		Verifica + validazione.

		\subsection{QUALITÀ} \index{Qualità} \label{qualita} %V.I
		L’insieme delle proprietà e delle caratteristiche di un prodotto che conferiscono ad esso la capacità di soddisfare esigenze espresse o implicite del cliente.
		%Serve per emancipare.
		Si vede chi fa (sulla base di qualche riferimento), chi usa (almeno il minimo obbligatorio) e chi valuta (etichettano).
		La qualità è \textit{di prodotto} e \textit{di processo}.\\
		Per quel che riguarda la \textit{qualità di prodotto} esistono:
		\begin{itemize}
			\item \underline{\hyperref[sistemadiqualita]{Sistema di Qualità}}: è la struttura organizzativa e le risorse messe in atto per perseguire la qualità, in cui c'è la supervisione di miglioramento continuo
			\item \underline{\hyperref[pianoqualifica]{Piano della Qualità}}
			\item \underline{\hyperref[controlloqualita]{Controllo di qualità}}
		\end{itemize}
		%Il modello di sviluppo muove il ciclo. Vogliamo vedere quanto abbiamo consumato per raggiungere quella quantità. \\
		La qualità è un poliedro: ha tante facce ed è in stretta correlazione a \underline{\hyperref[efficacia]{efficacia}} e valutazione.
		Riguardo alla \textit{qualità di processo} bisogna definire il \underline{\hyperref[processo]{processo}} per controllarlo e migliorarne efficacia, efficienza ed esperienza. Usare inoltre buone metriche e strumenti di valutazione. Ci riferiamo qui ora allo \underline{\hyperref[standard]{standard}} ISO 9000. \\
		Esiste inoltre un \underline{\hyperref[manualequalita]{Manuale della qualità}} e degli strumenti di valutazione tra cui:
		\begin{itemize}
			\item \textbf{SPY}: \textit{SW Process Assessment and Improvement} dà una valutazione oggettiva dei processi di una organizzazione per darne un giudizio di maturità e individuare azioni migliorative
			\item \textbf{\underline{\hyperref[cmmi]{CMMI}}}
			\item \textbf{\underline{\hyperref[15504]{ISO 15504}}}
		\end{itemize}
