\newpage
	\section{V} \label{sec:V}

		\subsection{VALIDARE} \index{Validare} \label{validare}
		\textit{``Did I build the right system?''}. \\
		Ha come obiettivo l'accertarsi che il prodotto sia conforme alle attese, infatti si fa sull'esito di sviluppo.
		Prevede \underline{\hyperref[testsistema]{test di sistema}} e \underline{\hyperref[collaudo]{collaudo}}.

		\subsection{VALUTAZIONE}
		Determinazione del valore da assegnare a cose o fatti ai fini di un giudizio.
		Varia in base ai destinatari perché hanno diverse aspettative. \\
		Notare che la \underline{\hyperref[qualita]{qualità}} fa da appoggio per la valutazione.

		\subsection{VERIFICARE} \index{Verificare} \label{verificare}
		\textit{``Did I build the system right?''}. \\
		Accertare che l'esecuzione delle attività di processi svolti nella \underline{\hyperref[fase]{fase}} in esame non abbia introdotto errori nel prodotto, infatti va fatta rivolta ai processi. \\
		Esistono diverse forme di verifica:	%set verifica e validazione - introd
		\begin{itemize}
			\item \textbf{Dinamica}: richiede l'esecuzione del programma e si fa tramite \underline{\hyperref[test]{test}}
			\item \textbf{Statica}: non richiede l'esecuzione del prodotto. La più semplice e umana è la lettura.
			Ci sono due modi di lettura:
			\begin{itemize}
				\item \underline{\hyperref[walkthrough]{Waltkthrough}}
				\item \underline{\hyperref[inspection]{Inspection}}: controllo mirato a cose chiamate \textit{lista di controllo}
			\end{itemize}
		\end{itemize}
		La verifica serve per scovare i problemi e risolverli tempestivamente tramite \textit{\underline{\hyperref[problemsolution]{problem solution}}}.

		\begin{figure}[H]
			\centering
			\includegraphics[width=0.6\textwidth]{img/v}
			\caption{Verifica e validazione nello sviluppo.}
			\label{V}
		\end{figure}
		%13 dicembre - verifica e validazione: analisi statica
		La programmazione non deve costituire ostacolo alla verifica, anche se spesso lo fa.
		L'obiettivo è quindi facilitare la verifica, sebbene esista anche il conflitto tra essere economici (poco) e sapere cosa accade (tanto). \\
		\textbf{Morale}: serve uno standard di verifica che si protrae in lungo, rendendo agevole la manutenzione, e che eviti assolutamente la verifica retrospettiva (ovvero, ho fatto e ora rivedo cos'ho fatto sistemando quello che ho sbagliato).
		Questo perchè il costo di rilevazione e correzione di errori cresce con l’avanzare dello sviluppo.	%V e V: analisi dinamica 20 dicembre %QUI?

		\begin{figure}[H]
			\centering
			\includegraphics[width=0.6\textwidth]{img/costi}
			\caption{Costo di correzione di errori.}
		\end{figure}

		Un buon approccio è quindi verificare tramite \underline{\hyperref[byconstruction]{correttezza per costruzione}}.
		È inoltre importante e di nostro interesse scrivere \underline{\hyperref[programmiverificabili]{programmi verificabili}}. \\
		\textbf{NB}: la verifica si può parallelizzare.


		\subsection{VERIFICATORE} \index{Verificatore} \label{verificatore}
		È uno dei \underline{\hyperref[ruoli]{ruoli}} in un progetto.
		Sono presenti per l’intera durata del progetto.
		Hanno capacità di giudizio e di relazione, oltre ad avere competenze tecniche, esperienza professionale e conoscenza	delle norme. \\


		\subsection{VERSIONE} \index{Versione} \label{versione}
		È un'istanza di un determinato \textit{configuration item}. Una versione è un'istanza di un configuration item diversa dalla precedente.
		%Software che è in fase di sviluppo. Gli avanzamenti possono a loro volta contenere altri avanzamenti. (Ci sono versioni per ogni parte di una baseline).
